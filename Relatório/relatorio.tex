%Preâmbulo
\documentclass[a4paper,12pt,titlepage,portuguese]{article}

\usepackage[portuguese]{babel}
\usepackage[utf8]{inputenc}
\usepackage[T1]{fontenc}
\usepackage{hyperref,textgreek}

\makeatletter
  \newcommand{\nop}[1]{\Hy@raisedlink{\hypertarget{#1}{}}}
\makeatother

\title{Relatório do Trabalho Prático de Sistemas Operativos\\[1cm]\large{\textbf{Backup-Eficiente}}\\\large{2015/2016-MIEI}}

\author{João Martins (A68646) \\João Pereira (A75273)}

\date{\today}
%Fim do Preâmbulo
\begin{document}

\pagenumbering{Alph}
\maketitle

\begin{abstract}
\pagenumbering{gobble}
O trabalho desenvolvido consiste num sistema de cópias de segurança de ficheiros capaz de realizar operações de \emph{backup} e restauro de ficheiros, bem como de remoção de ficheiros de cópia de segurança e \emph{garbage collection}. A segue uma arquitetura de cliente-servidor, dado que as várias opera em que um ou mais programas \texttt{cliente} enviam (através de um \emph{pipe} com nome) pedidos de  para um servidor que executa os pedidos e responde com os sinais \texttt{SIGUSR1} ou \texttt{SIGUSR2} de forma a indicar o sucess ou insucesso de cada operação, respetivamente. para indicar se a opera
\end{abstract}

\pagenumbering{roman}
\tableofcontents

\newpage

\pagenumbering{arabic}

\section{Aspetos gerais às várias tarefas}


\section{Conclusão}

\end{document}