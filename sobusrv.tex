\section{sobusrv}
\subsection{Utilização}
O servidor é iniciado fazendo apenas \texttt{sobusrv} numa shell. Este comando ignora quaisquer flags e argumentos passados e fica a correr em segundo plano, à 
espera que um cliente lhe envie comandos.
\subsection{Implementação}
%\subsubsection{main}
Quando o servidor é iniciado, é feita a inicialiazção das variáveis com os caminhos de ficheiros que serão utilizados em todo o programa:
\begin{enumerate}
\item backup_path - \$HOME/.Backup/
\item metadata_path -  \$HOME/.Backup/metadata/
\item data_path - \$HOME/.Backup/data/   
\item fifo_path - \$HOME/.Backup/fifo
\item logfile_path - \$HOME/.Backup/log.txt
\end{enumerate}
	
Imediatamente a seguir, é aberto o ficheiro \texttt{log.txt} para guardar o log das operações realizadas pelo servidor. Este ficheiro vai estar aberto para todos os 
subprocessos do processo inicial, de forma a que todos eles possam registar a operação que executam. 

Para podermos "desocupar" a shell, i.e. executar o processo em 
background, faz-se um fork em que o pai termina e o filho continua a executar. O programa executa ainda mais um fork em que o filho abre a ponta de escrita 
do fifo que se encontra na raíz do backup e depois invoca o pause, de forma a ficar adormecido.  
Se não fizessemos este fork, quando todos os processos fechassem a ponta de escrita do write (por ex: quando todos os comandos enviados pelo 
\texttt{sobucli} terminam), o processo "principal" do \texttt{sobusrv} que gere a crição de outros processos iria entrar em espera ativa porque não ia 
bloquear nos \texttt{reads} da FIFO, resultando numa performance deplorável. O processo pai continua a execução e cria 5 processos em que cada um lê um 
comando da FIFO e começa a sua execução através da invocação da função \texttt{setupComand()}. O processo pai fica à espera que os filhos terminem e sempre que algum
morre, é criado um novo processo que lê um comando da FIFO. Desta forma, asseguramos que, no máximo, temos 5 operações em simultâneo uma vez que só lemos um
novo comando quando outro termina. 

A função \texttt{setupComando()}, como foi dito anteriormente, lê um comando do FIFO e começa a sua execução. Mais detalhadamente, é lida uma estrutura 
Comando do FIFO e se a leitura tiver sido bem sucedida, é executado o comando nela guardado através da função \texttt{execComando()}. Finda a execução do comando,
é enviado para o processo cliente que escreveu o Comando na FIFO um sinal SIGUSR1 se o comando tiver sido executado com sucesso e SIGUSR2 se tiver havido algum
erro.

A função \texttt{execComando()} serve unicamente para invocar a operação correta(\texttt{backup()}, \texttt{restore()}, \texttt{gc()} ou \texttt{delete()})
com os argumentos necessários, se for o caso.

As operações suportadas estão implementadas da seguinte maneira:
%falar do sha1sum
\subsubsection{backup(char * filepath)}
Dado um caminho absoluto de um ficheiro, calculamos o \texttt{sha1sum} e inserimos o um \texttt{symlink} com o nome do ficheiro a guardar e que aponta
para \texttt{data/sum.gz}, em que \texttt{sum} é o resultado da função \texttt{sha1sum}. Se já existir um \texttt{symlink} com esse nome na \texttt{data},
o programa termina: definimos que para esses casos vamos manter o 1º ficheiro guardado. Caso contrário, escrevemos num ficheiro cujo nome é o nome do ficheiro a 
guardar iniciado por um caracter não-imprimível(de forma a evitar colisões de nomes). Esse ficheiro contém o path completo para 

 verificamos na pasta \texttt{data} se o ficheiro já foi guardado;

\subsubsection{restore(char * filename)}

\subsubsection{delete(char * filename)}

\subsubsection{gc()}

\subsection{frestore(char * absolute_path_folder)}
%Indicar que nao está a funcionar

%explicar funções auxiliares
