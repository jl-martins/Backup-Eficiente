\section{sobusrv}
\subsection{Utilização}
O servidor é iniciado introduzindo apenas \texttt{sobusrv} numa shell. Este comando ignora quaisquer flags e argumentos passados e fica a correr em segundo plano, à 
espera que um ou mais clientes lhe enviem comandos.
\subsection{Implementação}
Quando o servidor é iniciado, é feita a inicialiazção das variáveis com os caminhos de ficheiros que serão utilizados em todo o programa:
\begin{enumerate}
\item backup\_path - \$HOME/.Backup/
\item metadata\_path -  \$HOME/.Backup/metadata/
\item data\_path - \$HOME/.Backup/data/   
\item fifo\_path - \$HOME/.Backup/fifo
\item logfile\_path - \$HOME/.Backup/log.txt
\end{enumerate}
	
Imediatamente a seguir, é aberto o ficheiro \texttt{log.txt} para guardar o log das operações realizadas pelo servidor. Este ficheiro fica aberto para todos os 
subprocessos do processo inicial, de forma a que todos eles possam registar a operação que executam. 

Para podermos "desocupar" a shell, i.e. executar o processo em 
background, faz-se um fork em que o pai termina e o filho continua a executar. Posto isto, o programa executa ainda mais um fork em que o filho abre a ponta de escrita 
do fifo que se encontra na raíz do backup e depois invoca o pause, de forma a ficar adormecido.  
Se não fizessémos este fork, quando todos os processos fechassem a ponta de escrita do fifo (por ex: quando todos os comandos enviados pelo 
\texttt{sobucli} terminam), o processo "principal" do \texttt{sobusrv} que gere a crição de outros processos iria entrar em espera ativa porque não ia 
bloquear nos \texttt{reads} da FIFO, resultando numa performance deplorável. 

O processo pai continua a execução e cria 5 processos em que cada um lê um 
comando da FIFO e começa a sua execução através da invocação da função \texttt{setupComand()}. O processo pai fica à espera que os filhos terminem e sempre que algum
morre, é criado um novo processo que lê um comando da FIFO. Desta forma, asseguramos que, no máximo, temos 5 operações em simultâneo uma vez que só lemos um
novo comando quando outro termina. 

A função \texttt{setupComando()}, como foi dito anteriormente, lê um comando do FIFO e começa a sua execução. Mais detalhadamente, é lida uma estrutura 
Comando do FIFO e se a leitura tiver sido bem sucedida, é executado o comando nela guardado através da função \texttt{execComando()}. Finda a execução do comando,
é enviado para o processo cliente que escreveu o Comando na FIFO um sinal SIGUSR1 se o comando tiver sido executado com sucesso e SIGUSR2 se tiver havido algum
erro.

A função \texttt{execComando()} serve unicamente para invocar a operação correta(\texttt{backup()}, \texttt{restore()}, \texttt{frestore}, \texttt{gc()} ou \texttt{delete()})
com os argumentos necessários, se for o caso.

As operações suportadas estão implementadas, através de funções com o nome respetivo, da seguinte maneira:

\subsubsection{backup(char * filepath)}
Dado um caminho absoluto de um ficheiro, calculamos o \texttt{sha1sum} através da função \texttt{sha1sum()} e inserimos um \texttt{symlink} com o nome do ficheiro a guardar que aponta
para \texttt{data/sum.gz}, em que \texttt{sum} é o resultado da função \texttt{sha1sum()}. Se já existir um \texttt{symlink} com esse nome na \texttt{data},
o programa termina: definimos que para esses casos vamos manter o 1º ficheiro guardado. Caso contrário, escrevemos num ficheiro cujo nome é o nome do ficheiro a 
guardar iniciado por um caracter não-imprimível(de forma a evitar colisões de nomes). Esse ficheiro contém o path completo para o ficheiro que foi guardado.
Por fim, verificamos se exite na pasta \texttt{data} um arquivo cujo nome é o sha1sum do ficheiro sufixado de '.gz' e se não existir, fazemos o zip invocando a função \texttt{zipFile}.

\subsubsection{restore(char * filename)}
Dado o nome(e não o caminho absoluto) de um ficheiro, verificamos se esse ficheiro já foi guardado, bastando para tal ver se existe um ficheiro com
o nome do ficheiro a restaurar na pasta metadata(no caso de existir, corresponde ao symlink para o ficheiro nos dados). Se o ficheiro não tiver sido guardado, o programa termina.
Caso contrário, tentamos abrir o ficheiro que contém o path original do ficheiro e, em caso de sucesso, fazemos o unzip para esse local do ficheiro 
armazenado em data. Note-se que antes de restaurarmos o ficheiro, reconstruimos o path até ao local para onde o ficheiro vai ser restaurado
fazendo um fork e no filho:
\begin{verbatim}
execlp("mkdir", "mkdir", "-p", path, NULL);
\end{verbatim}

O pai faz um \texttt{wait} para assegurar que o caminho é criado antes de ele tentar restaurar o ficheiro.

\subsubsection{delete(char * filename)}
Dado o nome de um ficheiro(e não o seu caminho), verifica se existe um ficheiro com esse nome na pasta \texttt{metadata}. Se não existir, então é
devolvido 0 a indicar sucesso - se um ficheiro não existe, o objetivo de apagar o ficheiro é trivialmente alcançado.
Caso contrário, tentamos apagar o ficheiro que contem o path e o symlink através da \texttt{unlink}. Se não
conseguirmos apagar um desses ficheiros, é devolvido 1 a sinalizar a ocorrência de um erro. Caso contrário, é devolvido 0.
Esta função não apaga conteúdo na pasta \texttt{data}. Esse trabalho é deixado a cargo do comando gc.

\subsubsection{gc()}
Para cada ficheiro na pasta backup, se existir algum \texttt{symlink} para o mesmo, mantemos o ficheiro. Caso contrário, apagamos esse ficheiro.
Para percorrer os ficheiros da pasta de backup utilizamos a função \texttt{opendir} e \texttt{readdir}. Para verificarmos se um ficheiro tem algum symlink
na pasta \texttt{metadata} usamos a função \texttt{temLink()}, que dado o caminho absoluto para um ficheiro, devolve 1 se existe um \texttt{symlink} em \texttt{metadata} para ele e 0 caso contrário.

\subsubsection{frestore(char * absolute\_path\_folder)}
Dado um caminho absoluto de uma pasta, restaura a pasta e todos os ficheiros dessa pasta. 
O programa percorre todos os ficheiros guardados na pasta \texttt{metadata} e, sempre que encontra um ficheiro que guarda o path de um ficheiro guardado
(i.e. sempre que encontra um ficheiro cujo nome começa pelo caracter não imprimivel que definimos),
lê o conteúdo desse ficheiro(que corresponde ao caminho de um ficheiro) para um buffer e se o caminho absoluto da pasta passado como argumento for um prefixo 
do caminho que está no buffer(verificado com a função \texttt{strstr}, quer dizer que esse ficheiro estava na pasta que pretendemos restaurar e
 por isso restauramos o ficheiro com a restore.

\subsubsection{ sha1sum(char * filepath) - função auxiliar}
Dado o caminho absoluto para um ficheiro, devolve uma string com o seu sha1sum. Esta função cria um pipe anónimo e depois faz um \texttt{fork}. No processo filho,
fecha-se a ponta de leitura do pipe e depois, através do \texttt{dup}, redireciona o \texttt{stdout} para o pipe criado. A seguir, invoca a \texttt{execlp} 
de forma a executar o programa \texttt{sha1sum} e escreve o seu resultado no pipe anónimo. No pai, fecha-se a ponta de escrita do pipe.
O pai espera que o filho termine com um \texttt(wait) e verifica 
o código de saída do filho. Se este não indicar nenhum erro, o conteúdo do pipe é lido para um buffer que foi alocado dinamicamente e o endereço desse buffer
é devolvido.
\subsubsection{ zipFile(char * filepath, char * newFile, int opcao) - função auxiliar}
Dados dois caminhos absolutos, indicados no título como filepath e newFile, esta função faz zip/unzip do ficheiro especificado em filepath e coloca o resultado
no caminho indicado por newFile, dependendo do valor \texttt{opcao}. Os valores possíveis do argumento \texttt{opcao} são \texttt{ZIP} e \texttt{UNZIP}.
Quando é invocada, esta função cria um pipe anónimo e depois faz \texttt{fork}. O filho redireciona o \texttt{stdout} para a ponta de escrita do pipe e depois faz
um \texttt{exec} para executar o \texttt{gzip} com a opção \texttt{-c} para imprimir o resultado da compressão no \texttt{stdout}. O processo pai entra num loop
de leitura do contúdo do pipe e escrita no ficheiro pretendido. Quando o processo filho terminar e o fifo 'esvaziar', o pai sai do loop e a função termina.

\subsubsection{ temLink(char * filepath) - função auxiliar}
Dado um caminho absoluto para um ficheiro, esta função devolve 1 se existe um link para esse ficheiro na pasta \texttt{metadata} e 0 caso contrário.
Esta verificação é feita percorrendo os links da pasta \texttt{metadata} e se a expansão do caminho apontado por um link(através da \texttt{readlink}) 
resultar num caminho igual ao que é passado como argumento, então encontramos um link para esse ficheiro e devolvemos 1. Se nenhum link na pasta verificar 
esta condição, devolvemos 0;

\section{Melhorias possíveis}
Com o avançar do projeto apercebemo-nos que seria melhor implementar no processo "principal" do \texttt{sobusrv} 
(aquele que é responsável por criar processos que executam comandos) um gestor de ficheiros abertos que consistia num buffer de 
de strings com o nome dos ficheiros abertos nos processos filho de forma a evitar operações simultaneas no mesmo ficheiro. 
Desta forma, tiravamos a responsabilidade aos utilizadores de não executar comandos que acedam aos mesmos ficheiros em simultâneo.

No entanto, tal operação exigia uma reestrururação relativamente extensa do programa do servidor e tal seria impraticável na fase avançada
de desenvolvimento em que nos encontravamos.
