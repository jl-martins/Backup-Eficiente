%Preâmbulo
\documentclass[a4paper,12pt,titlepage,portuguese]{article}

\usepackage[portuguese]{babel}
\usepackage[utf8]{inputenc}
\usepackage[T1]{fontenc}
\usepackage{hyperref,textgreek}

\makeatletter
  \newcommand{\nop}[1]{\Hy@raisedlink{\hypertarget{#1}{}}}
\makeatother

\title{Relatório do Trabalho Prático de Sistemas Operativos\\[1cm]\large{\textbf{Backup-Eficiente}}\\\large{2015/2016-MIEI}}

\author{João Martins (A68646) \\João Pereira (A75273)}

\date{\today}
%Fim do Preâmbulo
\begin{document}

\pagenumbering{Alph}
\maketitle

\begin{abstract}
\pagenumbering{gobble}
O trabalho desenvolvido consiste num sistema de cópias de segurança de ficheiros escrito em \texttt{C}, que segue uma arquitetura cliente/servidor e suporta operações de \emph{backup} e restauro de ficheiros, bem como de remoção de ficheiros de cópia de segurança e \emph{garbage collection}. Todas as operações são executadas num programa servidor, a pedido de um ou mais programas cliente que poderão estar a executar em simultâneo, desde que estejam na mesma conta de utilizador que o servidor. Cada instância do programa cliente (\texttt{sobucli}) envia um comando para o servidor (\texttt{sobusrv}) através de um fifo (\emph{named pipe}) situado na raiz do backup. O sucesso/insucesso a executar um pedido de um cliente é indicado pelo servidor através dos sinais \texttt{SIGUSR1} (sucesso) e \texttt{SIGUSR2} (insucesso).

\end{abstract}

\pagenumbering{roman}
\tableofcontents

\newpage

\pagenumbering{arabic}

\section{Configuração}

\subsection{Instruções}

\subsubsection{Instalação}

Para instalar o sistema de cópias de segurança:
	\begin{enumerate}
		\item Descomprimir o ficheiro \texttt{SO\_Grupo3.zip};
		\item Invocar o comando \texttt{make install} e introduzir as credenciais do \emph{superuser} quando pedidas;
	\end{enumerate}
\textbf{Nota:} é necessário introduzir as credenciais do \emph{superuser} para mover os executáveis para a diretoria \texttt{/usr/bin}, permitindo assim que seja possível invocar \texttt{sobucli} e \texttt{sobusrv} sem especificar o caminho absoluto dos executáveis.

\subsubsection{Desinstalação}

Para desinstalar o sistema de cópias de segurança:
	\begin{enumerate}
		\item Invocar o comando \texttt{make unninstall} na diretoria que tem a \texttt{Makefile} do projeto e introduzir as credenciais do \emph{superuser} quando pedidas;
		\item Optar por apagar ou não apagar os dados dos \emph{backups} antigos quando questionado;
	\end{enumerate}
\textbf{Nota:} as credenciais de \emph{superuser} são necessárias apenas para remover \texttt{sobucli} e \texttt{sobusrv} da diretoria \texttt{/usr/bin}.

\newpage

\subsection{Makefile e scripts}

\subsubsection{Makefile}
Apresenta-se a \texttt{Makefile} utilizada pelo grupo para gerar os executáveis, invocar os scripts de instalação/desinstalação (\texttt{make install} e \texttt{make unninstall}, respetivamente) e gerar o relatório \\(\texttt{make relatorio}):
	\begin{verbatim}
	CFLAGS = -Wall -Wextra -O2
	TARGET_ARCH = -march=native

	all: sobucli sobusrv

	.PHONY: all install unninstall relatorio clean

	install: all
		bash install.sh

	unninstall:
		bash unninstall.sh

	relatorio:
		cd Relatorio; pdflatex relatorio.tex

	debug: CFLAGS = -Wall -Wextra -g
	debug: all

	sobucli: sobucli.o readln.o comando.o comando.h
		$(LINK.c) $^ $(OUTPUT_OPTION)

	sobusrv: sobusrv.o comando.o
		$(LINK.c) $^ $(OUTPUT_OPTION)

	clean:
		$(RM) sobucli sobusrv
		$(RM) *.o
		$(RM) Relatorio/relatorio.{aux,log,out,toc}
	\end{verbatim}

\newpage
\subsubsection{install.sh}
O \emph{script} de instalação \texttt{install.sh}:
	\begin{enumerate}
		\item Cria a diretoria \texttt{\$HOME/.Backup/} (raiz do \emph{backup});
		\item Cria as pastas \texttt{data} e \texttt{metadata} na raiz do \emph{backup};
		\item Invoca \texttt{mkfifo} para criar , na raiz do \emph{backup}, o \texttt{fifo} o utilizado para os clientes enviarem comandos ao servidor as pastas \texttt{data} e \texttt{metadata};
		\item Move os executáveis \texttt{sobucli} e \texttt{sobusrv} para a diretoria \texttt{/usr/bin};
	\end{enumerate}
\textbf{Nota:} se algumas das pastas/ficheiros a criar já existir, a instalação prossegue sem causar quaisquer problemas. 

Código de \texttt{install.sh}:
	\begin{verbatim}
	mkdir -p $HOME/.Backup/data
	mkdir -p $HOME/.Backup/metadata
	if [ ! -f "$HOME fifo" ]; then
		mkfifo $HOME/.Backup/fifo -m 0666
	fi
	sudo mv -t /usr/bin sobucli sobusrv
	\end{verbatim}

\subsubsection{unninstall.sh}
O script de desinstalação:
	\begin{enumerate}
		\item Remove o \emph{log} de erros de \texttt{sobusrv};
		\item Remove \texttt{sobucli} e \texttt{sobusrv} de \texttt{/usr/bin};
		\item Pergunta ao utilizador se pretende remover os dados dos seus \emph{backups} e, caso o utilizador pretenda fazê-lo, remove a pasta \texttt{\$HOME/.Backup} e todos os seus conteúdos;
	\end{enumerate}

Código de \texttt{unninstall.sh}:
	\begin{verbatim}
	rm -f  $HOME/.Backup/fifo $HOME/.Backup/log.txt

	sudo rm -f  /usr/bin/sobucli /usr/bin/sobusrv 
	echo "Deseja apagar os dados de backups antigos?"
	select sn in "Sim" "Nao"; do
	    case $sn in
	        Sim ) rm -rf $HOME/.Backup; break;;
	        Nao ) exit;;
	    esac
	done
	\end{verbatim}

\section{Estruturação da raiz do backup}

\section{Comandos}

O grupo implementou os seguintes comandos:

	\begin{itemize}
		\item \texttt{backup} para realizar \emph{backup} de um ficheiro ou de uma diretoria;
		\item \texttt{restore} para restaurar ficheiros para o seu local original;
		\item \texttt{delete} para remover um ficheiro da cópia de segurança;
		\item \texttt{gc} para eliminar todos os ficheiros da diretoria \texttt{/data} que não estejam a ser utilizados por nenhuma das entradas em \texttt{/metadata};
		\item \texttt{frestore} (\emph{folder restore}) para restaurar uma diretoria;
	\end{itemize}

\subsection

\section{Conclusão}

\end{document}