%Preâmbulo
\documentclass[a4paper,12pt,titlepage,draft,portuguese]{article}

\usepackage[portuguese]{babel}
\usepackage[utf8]{inputenc}
\usepackage[T1]{fontenc}
\usepackage{hyperref,textgreek}

\makeatletter
  \newcommand{\nop}[1]{\Hy@raisedlink{\hypertarget{#1}{}}}
\makeatother

\title{Relatório do Trabalho Prático de Sistemas Operativos\\[1cm]\large{\textbf{Backup-Eficiente}}\\\large{2015/2016-MIEI}}

\author{João Martins (A68646) \\João Pereira (A75273)}

\date{\today}
%Fim do Preâmbulo
\begin{document}

\pagenumbering{Alph}
\maketitle

\begin{abstract}
\pagenumbering{gobble}
O trabalho desenvolvido consiste num sistema de cópias de segurança de ficheiros escrito em \texttt{C}, que segue uma arquitetura cliente/servidor e suporta operações de \emph{backup} e restauro de ficheiros\/diretorias, bem como de remoção de ficheiros de cópia de segurança e \emph{garbage collection}. Todas as operações são executadas num programa servidor, a pedido de um ou mais programas cliente que poderão estar a executar em simultâneo, desde que estejam na mesma conta de utilizador que o servidor. Cada instância do programa cliente (\texttt{sobucli}) envia um comando para o servidor (\texttt{sobusrv}) através de um fifo (\emph{named pipe}) situado na raiz do backup. O sucesso/insucesso a executar um pedido de um cliente é indicado pelo servidor através dos sinais \texttt{SIGUSR1} (sucesso) e \texttt{SIGUSR2} (insucesso).

\end{abstract}

\pagenumbering{roman}
\tableofcontents

\newpage

\pagenumbering{arabic}

\section{Configuração}

\subsection{Instruções}

\subsubsection{Instalação}

Para instalar o sistema de cópias de segurança:
	\begin{enumerate}
		\item Descomprimir o ficheiro \texttt{SO\_Grupo3.zip};
		\item Invocar o comando \texttt{make install} e introduzir as credenciais do \emph{superuser} quando pedidas;
	\end{enumerate}
\textbf{Nota:} é necessário introduzir as credenciais do \emph{superuser} para mover os executáveis para a diretoria \texttt{/usr/bin}, permitindo assim que seja possível invocar \texttt{sobucli} e \texttt{sobusrv} sem especificar o caminho absoluto dos executáveis.

\subsubsection{Desinstalação}

Para desinstalar o sistema de cópias de segurança:
	\begin{enumerate}
		\item Invocar o comando \texttt{make unninstall} na diretoria que tem a \texttt{Makefile} do projeto e introduzir as credenciais do \emph{superuser} quando pedidas;
		\item Optar por apagar ou não apagar os dados dos \emph{backups} antigos quando questionado;
	\end{enumerate}
\textbf{Nota:} as credenciais de \emph{superuser} são necessárias apenas para remover \texttt{sobucli} e \texttt{sobusrv} da diretoria \texttt{/usr/bin}.

\newpage

\subsection{Makefile e scripts}

\subsubsection{Makefile}
Apresenta-se a \texttt{Makefile} utilizada pelo grupo para gerar os executáveis, invocar os scripts de instalação/desinstalação (\texttt{make install} e \texttt{make unninstall}, respetivamente) e gerar o relatório (\texttt{make relatorio}):
	\begin{verbatim}
	CFLAGS = -Wall -Wextra -O2
	TARGET_ARCH = -march=native

	all: sobucli sobusrv

	.PHONY: all install unninstall relatorio clean

	install: all
		bash install.sh

	unninstall:
		bash unninstall.sh

	relatorio:
		cd Relatorio; pdflatex relatorio.tex

	debug: CFLAGS = -Wall -Wextra -g
	debug: all

	sobucli: sobucli.o readln.o comando.o comando.h
		$(LINK.c) $^ $(OUTPUT_OPTION)

	sobusrv: sobusrv.o comando.o
		$(LINK.c) $^ $(OUTPUT_OPTION)

	clean:
		$(RM) sobucli sobusrv
		$(RM) *.o
		$(RM) Relatorio/relatorio.{aux,log,out,toc}
	\end{verbatim}

\newpage
\subsubsection{install.sh}
O \emph{script} de instalação \texttt{install.sh}:
	\begin{enumerate}
		\item Cria a diretoria \texttt{\$HOME/.Backup/} (raiz do \emph{backup});
		\item Cria as pastas \texttt{data} e \texttt{metadata} na raiz do \emph{backup};
		\item Invoca \texttt{mkfifo} para criar, na raiz do \emph{backup}, o \texttt{fifo} o utilizado para os clientes enviarem comandos ao servidor as pastas \texttt{data} e \texttt{metadata};
		\item Move os executáveis \texttt{sobucli} e \texttt{sobusrv} para a diretoria \texttt{/usr/bin};
	\end{enumerate}
\textbf{Nota:} se algumas das pastas/ficheiros a criar já existir, a instalação prossegue sem causar quaisquer problemas. 
\\Código de \texttt{install.sh}:
	\begin{verbatim}
	mkdir -p $HOME/.Backup/data
	mkdir -p $HOME/.Backup/metadata
	if [ ! -p $HOME/.Backup/fifo ]; then
		mkfifo $HOME/.Backup/fifo -m 0666
	fi
	sudo mv -t /usr/bin sobucli sobusrv
	\end{verbatim}

\subsubsection{unninstall.sh}
O script de desinstalação:
	\begin{enumerate}
		\item Remove o \emph{log} de erros de \texttt{sobusrv};
		\item Remove \texttt{sobucli} e \texttt{sobusrv} de \texttt{/usr/bin};
		\item Pergunta ao utilizador se pretende remover os dados dos seus \emph{backups} e, caso o utilizador pretenda fazê-lo, remove a pasta \texttt{\$HOME/.Backup} e todos os seus conteúdos;
	\end{enumerate}
Código de \texttt{unninstall.sh}:
	\begin{verbatim}
	rm -f  $HOME/.Backup/fifo $HOME/.Backup/log.txt

	sudo rm -f  /usr/bin/sobucli /usr/bin/sobusrv 
	echo "Deseja apagar os dados de backups antigos?"
	select sn in "Sim" "Nao"; do
	    case $sn in
	        Sim ) rm -rf $HOME/.Backup; break;;
	        Nao ) exit;;
	    esac
	done
	\end{verbatim}

\section{Estruturação das diretorias do backup}

\subsection{metadata}
Em \texttt{metadata} temos, para todo o ficheiro \texttt{f} salvaguardado com o comando \texttt{backup}:
	\begin{itemize}
		\item Um \emph{link} simbólico para o conteúdo original comprimido de \texttt{f}. O nome deste \emph{link} é igual ao do ficheiro original;
		\item Um ficheiro de texto com uma única linha que indica o caminho absoluto original do ficheiro \texttt{f}, permitindo-nos assim a recuperação do ficheiro para o seu local original, aquando da invocação do comando \texttt{restore}. O nome deste ficheiro de texto não pode ser igual ao do \emph{link} simbólico referido em \texttt{1.}, pelo que optamos por, partindo do nome do \emph{link} simbólico, acrescentar o caratére \texttt{Unit Separator} (\texttt{ASCII} 31) ao início do nome do ficheiro de texto.
	\end{itemize}

\subsection{data}
A diretoria \texttt{/data/}, situada na raiz do \emph{backup}, armazena os ficheiros salvaguardados com o comando \texttt{backup} num ficheiro comprimido com o programa \texttt{.gzip}. O nome do ficheiro comprimido é o \texttt{digest} do conteúdo do ficheiro original (produzido pelo programa \texttt{sha1sum}), seguido de \texttt{.gz}.\\\\\textbf{Nota:} se o utilizador tiver vários ficheiros com o mesmo conteúdo e nomes diferentes, apenas guardamos o conteúdo comprimido do primeiro desses ficheiros a ser enviado para \emph{backup}. Para cada um dos restantes ficheiros criamos somente um \emph{link} em \texttt{metadata} que aponta para o conteúdo comprimido já existente em \texttt{/data} e armazenamos o caminho original de cada um deles, evitando assim a duplicação de dados.

\section{Operações suportadas}

O programa desenvolvido suporta operações de:

	\begin{itemize}
		\item \emph{Backup} de ficheiros\/diretorias;
		\item Recuperação de ficheiros\/diretorias ;
		\item Remoção de ficheiros de cópia de segurança;
		\item \emph{Garbage collection}, i.e.: remoção de todos os ficheiros da pasta \texttt{/data} que não têm qualquer link simbólico da pasta \texttt{/metadata} que os referencie;
	\end{itemize}

\subsection{backup}

Ao invocar o comando \texttt{sobucli backup}, o utilizador pode optar especificar tanto caminhos absolutos como caminhos relativos nos ficheiros passados como argumentos para o comando, uma vez que o programa \texttt{sobucli} tem a capacidade de expandir caminhos relativos para caminhos absolutos antes de enviar um comando para o servidor.

\textbf{Nota:} de forma a seguir o esquema de \emph{links} simbólicos apresentado na página 3 do enunciado (antes da secção de \texttt{Valorização}), optamos por atribuir a cada \emph{link} simbólico o nome do ficheiro original a que este diz respeito (sem guardar o caminho absoluto desse ficheiro no nome do \emph{link}). Este tipo de abordagem leva a que não seja possível fazer \emph{backup} de 2 ficheiros que estão em diretorias diferentes e têm o mesmo nome. Uma forma de contornar esta limitação seria guardar no nome de cada \emph{link} o caminho absoluto do ficheiro comprimido para que este aponta. A impossibilidade de utilizar \'/\' no nome do \emph{link} poderia ser contornada substituindo \'/\' por um caratére que possa ser utilizado no nome de ficheiros e que praticamente nunca seja introduzido pela maioria dos utilizadores, como por exemplo o caratére \texttt{Unit Separator} (\texttt{ASCII} 31). Outra solução seria ter em \texttt{/metadata} cada uma das pastas constituintes do caminho dos ficheiros de que se fez \emph{backup} (replicando de certa forma a hierarquia do sistema de ficheiros). A título de exemplo, suponhamos que é pedido o \emph{backup} de 2 ficheiros de nome igual e conteúdo diferente, cujos caminhos são \texttt{/home/user/Desktop/a.pdf} e \texttt{/home/user/Documents/a.pdf}. Na abordagem referida, teríamos em \texttt{/metadata} os caminhos \texttt{/home/joao/Desktop/a.pdf} e \texttt{/home/joao/Documents/a.pdf}, em que o último elemento de cada um desses caminhos é um \emph{link} simbólico para o conteúdo comprimido do respetivo ficheiro original.

\subsection{restore}

Na invocação do comando \texttt{sobucli restore}, o utilizador deverá introduzir apenas o nome dos ficheiros que pretende restaurar para o seu caminho original. Note-se que não há qualquer ambiguidade porque, tal como foi referido na subsecção anterior, depois de o utilizador fazer \emph{backup} de um dado ficheiro, não poderá fazer \emph{backup} de outros ficheiros com o mesmo nome que o primeiro. É também importante notar que o comando \texttt{sobucli restore} serve para restaurar ficheiros e não diretorias. Para restaurar diretorias o utilizador deverá invocar o comando \texttt{sobucli frestore} (\emph{folder restore}).

\textbf{Nota:} Os comandos \texttt{sobucli backup} e \texttt{sobucli restore} não deverão ser executados em simultâneo, uma vez que a ordem pela qual chegam ao servidor não é necessáriamente a ordem pela qual são introduzidos e o resultado obtido depende da ordem de execução dos comandos. Assim, se o utilizador optar por executar \texttt{sobucli backup} e \texttt{sobucli restore} em simultâneo, o comportamento é indefinido.

\subsection{frestore}

O comando \texttt{sobucli frestore} (\emph{folder restore}) permite restaurar diretorias. Como várias diretorias diferentes podem ter o mesmo nome, é necessário especificar em cada um dos argumentos de \texttt{sobucli frestore}, o caminho absoluto das diretorias que se pretende restaurar.

\subsection{delete}

O comando \texttt{sobucli delete} recebe uma lista de um ou mais \underline{ficheiros} e apaga as referências que possam existir em \texttt{metadata\/} para cada um desses ficheiros. Se não existir nenhum \emph{link} para um dado ficheiro passado para o comando \texttt{delete}, consideramos que o objetivo de apagar esse ficheiro é trivialmente cumprido, pelo que se considera que \texttt{delete} teve sucesso.

\subsection{garbace collector (gc)}

O comando \texttt{sobucli gc} não recebe quaisquer argumentos e é responsável por eliminar todos os ficheiros da diretoria \texttt{data\/} que não estão a ser usados por nenhuma entrada em \texttt{\/metadata}. Se o utilizador passar argumentos adicionais para \texttt{sobucli gc}, é impressa uma mensagem de erro a indicar que \texttt{sobucli gc} não recebe quaisquer argumentos.

\section{struct comando}

De forma a agrupar numa única estrutura os vários dados que o servidor necessita para executar um determinado comando, o grupo optou por definir a \texttt{struct command}, permitindo assim que a informação de cada comando enviado para servidor tenha um número fixo de bytes (dado por \texttt{sizeof(struct command)}), o que facilita o acesso à mesma. 

Cada comando que é enviado para \texttt{sobusrv} tem 3 campos:

	\begin{itemize}
		\item \texttt{pid} do processo que enviou o comando, de forma a que o servidor possa enviar o sinal que indica conclusão de uma operação ou erro ao processo que ficou encarregue por pedir essa operação;
		\item Caminho relativo\/absoluto dos ficheiros a utilizar no comando a executar. No caso do comando \texttt{sobucli gc} este campo é uma \emph{String} começada por \'\\0\';
		\item Abreviatura do comando a realizar (1º caratére do nome do comando); 
	\end{itemize}

	Segue-se a declaração da \texttt{struct comando}:

	\begin{verbatim}
	struct comando{
		pid_t pid;
		char filepath[MAX_PATH];
		char codigoComando;
	};
	\end{verbatim}
, onde \texttt{MAX\_PATH} é uma macro definida em \texttt{comando.h} como 2048.
\textbf{Nota:} \underline{não confundir} a macro \texttt{MAX\_PATH}, definida pelo grupo, com a macro \texttt{PATH\_MAX}, definida em \texttt{Linux} e \texttt{Mac OSX}. \texttt{PATH\_MAX} está definida na maioria dos sistemas operativos \texttt{Linux} como 4096, devido a este ser o tamanho máximo para o qual há garantia que uma invocação da \emph{system call} \texttt{write()} é atómica. A razão para o grupo limitar o tamanho máximo de cada \emph{path} a 2048 caratéres \texttt{ASCII} prende-se com a necessidade de, em cada comando enviado para o servidor, enviarmos informação adicional como o \emph{pid} do processo que enviou o comando e a abreviatura desse comando. Se o tamanho máximo de cada caminho fosse 4096 caratéres, a escrita adicional da abreviatura do comando e do \texttt{pid} do processo que o enviou iriam implicar um \texttt{write} de um número superior a 4096 caratéres, pelo que não teríamos garantias relativamente à atomicidade de cada escrita no \texttt{fifo} usado para comunicar como o servidor. Se as escritas no \texttt{fifo} não fossem atómicas, ao existir mais do que um processo \texttt{sobucli} a escrever no \emph{named pipe}, correríamos o risco de enviar partes de comandos diferentes intercaladas umas com as outras, corrompendo as mensagens que se pretende transmitir.

De forma a abstrair os detalhes de implementação da \texttt{struct comando} e ao mesmo tempo proteger os conteúdos da mesma, o grupo definiu o tipo \texttt{Comando} no ficheiro \texttt{comando.h} da seguinte forma:

	\begin{verbatim}
	typedef struct comando * Comando;
	\end{verbatim}

A criação\/destruição de comandos, bem como a obtenção cada um dos campos da \texttt{struct comando} é realizada com as seguintes funções:

	\begin{verbatim}
	/* Aloca memória para um comando. */
	Comando aloca_comando();
	/* 
	 * Aloca memória para um comando e inicializa-o com o o código de comando
	 * passado como 1º argumento e com o caminho de ficheiro passado como 2º
	 * argumento. O pid associado ao comando é o do processo que chama a função.
	 */
	Comando aloca_inicializa_comando(char codigoComando, char * caminho_ficheiro);
	/* Devolve o pid do processo que enviou o comando passado como argumento. */
	pid_t get_pid_comando(Comando cmd);
	/* Devolve o caminho do ficheiro associado ao comando passado como argumento. */
	char * get_filepath(Comando cmd);
	/* Devolve o código do comando passado como argumento. */
	char get_codigoComando(Comando cmd);
	/* Devolve o número de bytes que um comando ocupa em memória. */
	ssize_t tamanhoComando();
	\end{verbatim}

\section{sobucli}

\subsection{Utilização}

\texttt{sobucli} - envia para \texttt{sobusrv} comandos de cópia, restauro e remoção de ficheiros, bem como de \emph{garbage collection}.
\\\\\texttt{sobucli comando [ficheiros]...}

\subsection{Implementação}

\subsubsection{Validação de comandos}

O envio de comandos que sabemos logo à partida que são inválidos iria ocupar desnecessariamente \texttt{sobusrv}, pelo que optamos por realizar logo à partida todas as validações que o cliente possa fazer. Assim, mal uma instância de \texttt{sobucli} começa a executar, verificamos:

	\begin{enumerate}
		\item se \texttt{argc} é superior a 1 (i.e.: se o utilizador introduziu um comando);
		\item se, nos casos em que o comando (\texttt{argv[1]}) é \texttt{gc}, \texttt{argc} é igual a 2 (isto porque \texttt{gc} não recebe quaisquer argumentos);
		\item se, nos casos em que o comando introduzido é \texttt{backup}, \texttt{restore}, \texttt{delete} ou \texttt{frestore}, temos \texttt{argc >= 3}, i.e.: se foram especificados ficheiros à direita do comando introduzido;
	\end{enumerate}
Se alguma das condições acima referidas falhar, o programa cliente imprime no \texttt{stderr} a mensagem de erro adequada à pré-condição que falhou e termina a execução com \texttt{\_exit(1)} sem chegar a abrir o \texttt{fifo} para escrita.

\subsubsection{Abreviaturas de comandos} \label{Abreviaturas de comandos}

Atentando no facto de todos os comandos implementados começarem com uma inicial diferente, o grupo optou por abreviar cada comando à sua 1ª letra, tendo para esse fim criado a função \texttt{get\_cmd\_abbrev}:

	\begin{verbatim}
	/* 
 	 * Valida o comando introduzido com base em argc e 
 	 * argv e, se este for válido, devolve uma abreviatura 
 	 * do mesmo. A abreviatura devolvida nao é mais do que
 	 * a 1ª letra do comando. Se o comando ou o número de
 	 * argumentos forem inválidos, imprime uma mensagem de erro 
 	 * no stderr e  devolve o carater '\0'.
 	 */
	char get_cmd_abbrev(int argc, char* argv[]);
	\end{verbatim}
\texttt{get\_cmd\_abbrev} realiza cada uma das validações listadas em \ref{Abreviaturas de comandos} e se alguma falhar, apresenta uma mensagem de erro de acordo com a condição que falhou \texttt{stderr} e devolve \'\\0\'. Se nenhuma das validações falhar, é devolvido o primeiro caratére do comando introduzido. 

A utilização de abreviaturas de comandos permite que sempre que pretendemos testar qual é o comando a ser executado não tenhamos que invocar a função \texttt{strcmp}.

\subsubsection{Variáveis globais ao módulo sobucli}

No início de \texttt{sobucli.c} temos a declaração das seguintes variáveis globais:

	\begin{verbatim}
	static char cmd_abbrev = '\0';
	static char* last_file = NULL;
	static int nchild;
	\end{verbatim}
\texttt{cmd\_abbrev} guarda a abreviatura do comando que está a ser executado, ou \texttt{\'\\0\'} se ainda não foi lido um comando válido. Esta variável é global para que no interior do \texttt{sighandler} invocado para tratar \texttt{SIGUSR1} e \texttt{SIGUSR2} tenhamos informação relativamente ao comando que foi executado e possamos imprimir a mensagem de sucesso/erro adequada. 

\texttt{last\_file} é um apontador para o caminho absoluto\/relativo do último ficheiro tratado numa determinada operação. Esta variável também é global para que possa ser acedida \texttt{sighandler} de \texttt{SIGUSR1} e \texttt{SIGUSR2} e possamos imprimir o nome e/ou o caminho do ficheiro utilizado num dado pedido.

\texttt{nchild} permite controlar o número de processos filhos criado pelo programa cliente, uma vez que, para cada ficheiro da lista de ficheiros passada nos comandos \texttt{backup}, \texttt{restore} e \texttt{delete}, criamos um processo filho que fica responsável por enviar o seu \texttt{pid}, o comando a executar e o ficheiro\/caminho absoluto a usar na operação. Embora não tenha sido imposto um limite ao número de processos filho que uma instância de \texttt{sobucli} poderá criar, o grupo optou por impor um limite de 5 processos filho em cada instante. A forma como o número de processos filho é controlado será apresentada com mais detalhe na secção \ref{Envio de comandos}.

\subsubsection{Construção do caminho da fifo}

O caminho absoluto do \texttt{fifo} é obtido da seguinte forma:
	\begin{verbatim}
		char backup_path[MAX_PATH];
		(...)
		strcpy(backup_path, getenv("HOME"));
		strcat(backup_path, "/.Backup");
		strcat(backup_path, "/fifo");
	\end{verbatim}

\subsubsection{Caminhos absolutos dos argumentos}

O servidor (\texttt{sobusrv}) necessita de ter conhecimento do caminho absoluto dos ficheiros de que o cliente pretende fazer \emph{backup} para que seja possível restaurá-los para o seu caminho original. Assim, mesmo que o utilizador invoque o comando \texttt{sobucli backup} com um ou mais ficheiros para os quais especifica apenas um caminho relativo à sua \emph{working directory}, o programa \texttt{sobucli} trata de expandir cada um desses caminhos relativos para o respetivo caminho absoluto, através da invocação da função \texttt{realpath}, importada da \texttt{stdlib}. A utilização de \texttt{realpath} possibilita também a realização de verificações adicionais, uma vez que se o utilizador especificar um caminho inválido \texttt{realpath} devolve \texttt{NULL}, permitindo que o cliente apresente uma mensagem de erro e passe para o próximo argumento de \texttt{backup} sem ter que \"incomodar\" o servidor.

Sempre que \texttt{realpath} expande com sucesso o caminho absoluto de um ficheiro passado como argumento para o comando de \texttt{backup}, testamos se o caminho absoluto obtido diz respeito a um ficheiro ou a uma diretoria, recorrendo para isso às funções \texttt{is\_file} e \texttt{is\_dir}, definidas com o auxílio da \emph{system call} \texttt{stat} da seguinte forma:

	\begin{verbatim}
	int is_file(char* path){
		struct stat sb;

		if(stat(path, &sb) == -1){
			perror("stat");
			return 0;
		}
		return S_ISREG(sb.st_mode);
	}

	int is_dir(char* path){
		struct stat sb;

		if(stat(path, &sb) == -1){
			perror("stat");
			return 0;
		}
		return S_ISDIR(sb.st_mode);
	}
	\end{verbatim}


\subsubsection{Envio de comandos} \label{Envio de comandos}

Se os argumentos de \texttt{sobucli} fossem enviados numa única linha para \texttt{sobusrv} haveria o risco de o número de bytes enviados exceder \texttt{PIPE\_BUF} e nesse caso deixaríamos de ter garantias relativamente à atomicidade das escritas, levando a que quando são executadas várias instâncias de \texttt{sobucli} concorrentemente, partes de comandos de diferentes fossem intercaladas umas com as outras. Além disso, quando o servidor responde com \texttt{SIGUSR1} ou \texttt{SIGUSR2} de forma a indicar o sucesso ou insucesso de uma operação também é necessário saber a que operação este se refere e se um único processo enviasse todos os argumentos recebidos por \texttt{sobucli} numa única escrita, dificilmente se saberia a que ficheiro se refere cada um dos sinais recebidos. Assim, de forma a garantirmos a atomicidade das escritas e ao mesmo tempo termos uma forma de saber a que ficheiro diz respeito cada um dos sinais enviados por \texttt{sobusrv} (nas várias operações que não \texttt{gc}) optamos por, para cada ficheiro dos argumentos de um dado comando, criar um processo filho de \texttt{sobucli} que fica encarregue de enviar o comando que diz respeito a esse ficheiro. A título de exemplo, se \texttt{sobucli} fosse invocado da seguinte forma \texttt{sobucli backup a.pdf b.pdf}, o processo original de \texttt{sobucli} iria criar 2 processos filho, um para enviar uma \texttt{struct Comando} que tem o seu \texttt{pid}, a abreviatura do comando (\'b\') e o caminho absoluto do ficheiro \texttt{a.pdf} e outro para enviar outra \texttt{struct Comando} com o seu \texttt{pid}, o comando \'b\' e o caminho absoluto de \texttt{b.pdf}. Assim, os comandos de \emph{backup} de um dado ficheiro, bem como \texttt{restore}, \texttt{frestore} e \texttt{delete} são enviados com através a função \texttt{send_cmd}, definida do seguinte modo:
	
	\begin{verbatim}
	void send_cmd(int fifo_fd, char* arg_path, char* full_path){
		Comando cmd;

		if(nchild >= 5){ /* limita o número de processos filho */
			wait(NULL);
			--nchild;
		}

		switch(fork()){
			case 0: /* processo filho */
				last_file = (arg_path != NULL) ? arg_path : full_path;
				if(full_path)
					cmd = aloca_inicializa_comando(cmd_abbrev, full_path);
				else
					cmd = aloca_inicializa_comando(cmd_abbrev, arg_path);
				
				if(write(fifo_fd, cmd, tamanhoComando()) != tamanhoComando())
					PERROR_AND_EXIT("write")
				
				close(fifo_fd);
				free(cmd);
				pause();
				_exit(1);
			case -1:
				perror("fork");
				_exit(1);
		}
	}
	\end{verbatim}

\texttt{fifo_fd} é o descritor do \texttt{fifo} por onde o comando será enviado, \texttt{arg\_path} é o caminho do ficheiro, tal como surge nos argumentos de \texttt{sobucli} e \texttt{full_path} é o caminho absoluto do ficheiro. No caso dos comandos de \texttt{backup} a função \texttt{send\_cmd} é sempre invocada com o \texttt{full\_path} do ficheiro a enviar para o servidor e \texttt{arg\_path} a \texttt{NULL}. No caso de \texttt{gc},  \texttt{arg\_path} é a string vazia \"\", de forma a indicar que o comando não tem argumentos. Para os restantes comandos é passado o \texttt{arg_path} e \texttt{full_path} é passado a \texttt{NULL}.

Também é importante notar que no início de cada invocação de \texttt{send\_cmd} verificamos o valor do variável global \texttt{nchild} e se este for maior ou igual a 5, o processo pai espera por um dos seus filhos antes de prosseguir a execução e fazer um \texttt{fork()}.

Chegando ao \texttt{switch}, o processo pai faz \texttt{fork()} e o processo filho guarda na variável global \texttt{last\_file} o caminho absoluto ou relativo (dependendo do comando a executar) do ficheiro a que o comando diz respeito (no caso de \texttt{gc} não é um ficheiro, mas sim uma \emph{String} vazia tal como já foi referido). Posto isto, se o processo \texttt{full\_path} for diferente de \texttt{NULL}, o processo filho cria um \texttt{Comando} com a variável global \texttt{cmd_abbrev} e com o argumento \texttt{full\_path}, se não cria um comando com \texttt{cmd\_abbrev} e \texttt{arg\_path}. Em seguida, este comando é escrito no \texttt{fifo} e em caso de sucesso, o processo filho fecha o descritor de escrita do \texttt{fifo}, liberta \texttt{cmd} e invoca \texttt{pause()} para aguardar pelo sinal do servidor a indicar conclusão ou erros na operação.

\subsubsection{Backup de diretorias}

Sempre que num comando de \emph{backup} um caminho absoluto remete para uma diretoria, executamos o programa \texttt{find}para obter os caminhos de cada ums dos ficheiros contidos nas subdiretorias da diretorias indicada e invocamos \texttt{send\_cmd()} com cada um dos ficheiros obtidos.

Como uma execução do programa \texttt{find} pode produzir uma grande quantidade de linhas, o grupo optou por definir a seguinte função:

	\begin{verbatim}
	int find(const char* path){
		pid_t p;
		int pipefd[2];
		int out_fd, status;

		if(pipe(pipefd) == -1)
			return -1;

		p = fork();
		if(p == 0){ /* filho */
			close(pipefd[0]);
			if(dup2(pipefd[1], 1) == -1)
				_exit(1);
			else
				close(pipefd[1]);

			execlp("find", "find", path, NULL);
			_exit(1);
		}
		else if(p == -1){
			close(pipefd[0]); close(pipefd[1]); out_fd = -1;
		}
		else{ /* pai */
			close(pipefd[1]);
			p = wait(&status);
			if(p != -1 && WIFEXITED(status) && WEXITSTATUS(status) == 0)
				out_fd = pipefd[0];
			else{ /* algo correu mal no processo filho */
				close(pipefd[0]);
				out_fd = -1;
			}
		}
		return out_fd;
	}
	\end{verbatim}
Tal como pode ser observado, o processo que invoca a função \texttt{find()} cria um pipe anónimo e faz \texttt{fork()}. Se o \texttt{fork()} tiver sucesso, o processo filho fecha o descritor de leitura do \emph{pipe} e redireciona o \texttt{standar output} para a ponta de escrita do \emph{pipe}. Uma vez feito este redirecionamento, o processo filho invoca a \emph{system call} \texttt{execl()} para executar a função \texttt{find} com o \emph{path} passado. Já o processo pai fecha a ponta de escrita do \emph{pipe} e espera pelo filho. Se o filho tiver terminado a sua execução com sucesso, o processo pai guarda em \texttt{out\_fd} o descritor de leitura do \emph{pipe} e devolve-o para a função chamadora, para que seja possível obter cada uma das linhas produzidas por \texttt{find}.

Para que a leitura de cada uma das linhas produzidas pela função \texttt{find} seja eficiente, optamos por usar a função \texttt{int readln(struct buffer_t *bufer, void **buf)} baseada no exercício 7. do guião 1. Nesta versão de \texttt{readln()}, cada invocação de \texttt{read} lê até 4096 bytes para um \emph{buffer} interno a \texttt{struct buffer\_t} e em cada chamada de \texttt{readln()} é escrita uma linha no buffer apontado por \texttt{buf}.

Como o programa \texttt{find} também pode incluir diretorias no seu resultado, testamos para cada linha lida, se esta é um ficheiro com a função \texttt{is\_file} (já apresentada acima) e se for, invocamos \texttt{send\_cmd()} para que o servidor faça \emph{backup} do ficheiro encontrado na diretoria de que estamos a fazer \emph{backup} ou nas suas subdiretorias. 

	

\texttt{arg_path} designa um caminho indicado por um argumento passado para \texttt{sobucli} por um processo filho de \texttt{sobucli} que guarda na sua instância da variável global \texttt{last_file} o caminho absoluto 

\section{Conclusão}

\end{document}